%You can delete all the comments after you have finished your document
%this sets up the defaults for the documents, 12pt font and A4 size. The article type sets this up as such as opposed to letter or memo.

%for the finer points LaTeX see https://en.wikibooks.org/wiki/LaTeX or http://tex.stackexchange.com/

\documentclass[12pt,a4paper]{article}
\usepackage{titlesec} %these are how we import packages, one helps set up footers and title layout
\usepackage{fancyhdr}

% !TEX TS-program = pdflatex
% !TEX encoding = UTF-8 Unicode
\usepackage[utf8]{inputenc} % set input encoding (not needed with XeLaTeX)

\usepackage{graphicx} % support the \includegraphics command and options
\graphicspath{ {./Images/}}

% \usepackage[parfill]{parskip} % Activate to begin paragraphs with an empty line rather than an indent

%%% PACKAGES
\usepackage{booktabs} % for much better looking tables
\usepackage{array} % for better arrays (eg matrices) in maths
\usepackage{paralist} % very flexible & customisable lists (eg. enumerate/itemize, etc.)
\usepackage{verbatim} % adds environment for commenting out blocks of text & for better verbatim
\usepackage{subfig} % make it possible to include more than one captioned figure/table in a single float
\usepackage[toc,page]{appendix}
\usepackage{titlesec}
% These packages are all incorporated in the memoir class to one degree or another...
\setcounter{secnumdepth}{4}

%header and footer settings
\pagestyle{fancyplain}
\fancyhf{}
\renewcommand{\headrulewidth}{0.5pt}
\renewcommand{\footrulewidth}{0.5pt}
\setlength{\headheight}{15pt}
\fancyhead[L]{Jonathan Mitchell - 40311730}
\fancyhead[R]{ SOC10101 Honours Project}
\fancyfoot[L]{}
\fancyfoot[C]{\thepage}

%set better section layout
\makeatletter
\renewcommand\subsection{\@startsection {subsection}{1}{2mm} % name, level, indent
	{3pt plus 2pt minus 1pt} % before skip
	{3pt plus 0pt} % after skip
	{\normalfont\bfseries}}
\makeatother
\makeatletter
\renewcommand\section{\@startsection {section}{1}{0mm} % name, level, indent
	{4pt plus 2pt minus 1pt} % before skip
	{4pt plus 0pt} % after skip
	{\bfseries}}
\makeatother

%this starts the document
\begin{document}
	
	%you can import other documents into your main one, these layout the Title and Declarations on its own page.
	%you might need to change these to \ if your on Microsoft Windows.
	\input{./Dissertation-Title.tex}
	\input{./Dissertation-Dec.tex}
	\pagebreak
	\input{./Dissertation-DP.tex}
	\pagebreak
	
	%LaTeX let you define the abstract separately so it wont get sucked into the main document.
	\begin{abstract}
		%Type stuff here
		This project is designed to explore the Microservice Architecture (MSA) to develop a flexible E-commerce web system. The architecture will be explored by researching and discussing various principles, tenets, architectures and technologies that can be used to develop web systems using MSA. Due to the architecture being relatively new; only being discussed and implemented in the last few years. There is no official consensus on how this architecture can be implemented. There are several suggestions and ideas available that are discussed and explored by members of the profession. These include suggestions such as: incorporating current principles like single-responsibility principle, that is part of the SOLID principle. Enforcing Domain Driven Design (DDD) to create bounded context for microservices to operate within and define where communication between microservices may be needed, automated testing and deployment pipelines and DevOps. 
		
		Modern technologies such as containerisation (Docker and Kubernetes) and the Cloud (Azure and Amazon Web Services(AWS)) are being suggested for use with MSA. These forms of Virtualisation aid in promoting a distributed system design - suggested for MSA. Using the Cloud for deployment purposes has contributed to the development of web-based systems. Integrated Development Environments (IDE's) such as Visual Studio, Eclipse and IntelliJ are available for this. 
		
	\end{abstract}
	\pagebreak
	
	\tableofcontents % is generated for you
	\newpage
	
	\listoftables
	%generated in same way as figures
	\newpage
	
	\listoffigures
	%you may have captions such as equations, listings etc they should all appear as required
	%these are done for you as long as you use \begin{figure}[placement settings] .. bla bla ... \end{figure}
	\newpage
	
	\section*{Acknowledgements}
	Insert acknowledgements here
	I would like to thank my supervisor, Xiaodong, for his advice and support with my dissertation. My Degus. My parents, Maya and Graham for listening to my rambling about my dissertation and attempts to explain it to them.
	\newpage
	\section{Chapter 1: Introduction}
	\subsection{Background}
	Traditionally, software applications/systems are developed using the Monolithic Architecture – a unified model of the design of a software program \cite{MonolithMargaret}. They are designed to be self-contained: components of the program are interconnected and interdependent. For this tightly coupled architecture to work, each component and dependent component’s must be present for the code to compiled or executed. If any component is missing, there is a high chance the program will not work correctly.
	The rapid increase of software systems has changed the shape of how businesses function in today’s world. As businesses have expanded and added new products/services, problems arose where the Monolithic code base would become difficult to maintain and any future changes could result in problems arising when implemented.
	For the past thirty years, the software industry has been moving every closer to a service-orientated approach \cite{SoADummies}. This evolution has resulted in a closer bond between businesses and IT. Instead of businesses making decisions controlled or constrained by software, they now make decision supported by software. 
	
	Service Orientated Architectures (SOA) were the first realisation of transforming monolithic systems into small building blocks – components – that work together to create applications that are easier to maintain and expand upon. A service is defined as: a function that is well-defined, self-contained, and does not depend on the context or state of other services \cite{ServiceArchiBarry}.
	Around seven years ago, at a workshop of architects in Venice, the participants saw a common architectural style they had all been recently exploring. The term “Microservices” was created \cite{MartinFowlersite}. It describes a particular way of designing software applications as suites of independently deployable services. That can be maintained and modified effectively to meet the demands of the business world in the present day.
	
	By describing each function, of a software system, as an individual service; they can be separated into their own components, contained within their own domain model. Doing this entails each service has a single responsibility to perform and no more. This would also entail each service to have its own database with which to manipulate data. There would be no shared relational database for the entire system. Although there may be a need for different services to communicate with each if they share specific information in order to maintain the validity of data their databases. The development of containerised software has increased the capability of polyglot programming. Were each Microservice can be written in a different language, and when deployed via containers, they combine, like building blocks, to form the software system. This form of modularity allows for each Microservice to function independently and improves the stability of the system as a whole - if one Microservice was to fail, it would not result in the entire system from failing. 
	
	With the recent boom of the internet, particularly in the last decade, Microservices are beginning to be used by retail companies with a large online presence such as: Amazon, Netflix and Ebay. Although retail is not the only sector to use this architecture: Uber, the guardian and Capital One. As the internet has proven to be an excellent platform to deploy systems. More specifically web-based systems.
	
	This project investigates and evaluates: designs, principles and technologies related to Microservices used too implement this architecture in developing an E-Commerce web system.
	\subsection{Motivation}
	why are you doing this? needed?
	
	\subsection{Aims, Objectives and Scope}
	The aim of this project is to explore how a microservice architecture (MSA) can be used to develop a flexible web based E-commerce system. Finding out the flexibility such an architecture provides and how this can be used to allow developers too maintain/update such systems in the fast-paced environment of today’s world. This project will concentrate on developing a web-based system as the internet has been one of the fastest growing resources available to businesses to expand and meet the ever growing demands of customers. 
	 
	These aims are achieved by examining the literature available on Microservices, the principles involved to enforce the Microservice Architecture and the technologies available to create this architecture for developing software to define research questions.
	A software development environment is then chosen to implement this architecture in a prototype system.
	Literature research will include: academic white papers, professional journals, lectures, on-line articles and software books. The collected/researched information will be used to answer the questions and provide the necessary understanding to develop a prototype using the microservice architecture. Following this, the project is evaluated, considering the original aims and objectives.
	
	As MSA is a relatively new architecture; there is still a broad view for successful implementation. Due to time and cost constraints, this project will focus on web-based design patterns and protocols. As well as software and hardware to design a web system. There will be a limit to the exploration of  applicable principles, design patterns, hardware and software available that cna be used to implement a microservice architecture.
	
	\subsection{Outline}
	This dissertation is structured as follows:
		\begin{itemize}\itemsep0pt
			\item \textbf{Chapter 1: Introduction}
			\\Introduces the topic of microservices, names the aims and objectives and outlines the scope and constraints for this project.
			\item \textbf{Chapter 2: Literature Review}
			\\Encompasses all the subjects and terms that are related/representative of a microservice architecture. It will go into detail about why Microservices are being used, why it is preferred over Service-orientated architecture and the technologies currently in use and those being developed. This chapter will also provide a critical and objective analysis of these subjects. 
			\item \textbf{Chapter 3: Project Planning}
			\\Provides an overview of the management of the project. With use of a Gantt chart. The methodologies used during the development cycle will also be described. With justification. This chapter will also include a description of the high level functionality needed for the project using the project management tool MoSCoW.
			\item \textbf{Chapter 4: Prototype} 
			\\ This chapter contains the following sections:
				\begin{itemize}
					\item \textbf{Analysis}
					\\This chapter will describe the requirements for the development of a prototype. And an analysis of the high-level design of the prototype. Including a diagram of the service interaction.
					\item \textbf{Design}
					\\This chapter will provide a detailed design of the structure of the prototype – UI designs, Class diagrams, E-R models etc.
					\item \textbf{Implementation}
					\\This chapter will detail the actual development of the prototype. Each staged will be documented. All issues/problems will also be documented along with the solutions. Also, providing a reference for all sources used. Screen shots of the development stages will be provided via appendices.
					\item \textbf{Testing}
					\\This chapter will consist of all testing documentation and testing conducted on the prototype. With a description of the testing methodologies used.
					\item \textbf{Evaluation}
					\\This chapter will provide a critical and objective analysis of the developed prototype. Providing a detailed description of its success/failure.
				\end{itemize}
			\item Chapter 5: Conclusions
			\\Discusses the project, how effective Microservices architecture is etc. And going forward what further research etc. will be done/required.
			\item Chapter 6: References
			\\Contains all references used throughout this document.
			\item Chapter 7: Appendices
			\\Houses all the appendices. This includes Gantt charts, UML diagrams, testing documentation and screen shots of: development progress at various stages and working prototype.
		\end{itemize}
	\subsection{Summary}
	This project encompasses the following principles, design patterns and technology:
	\begin{itemize}
		\item principle 1
		\item principle 2
		\item principle 3
		\item Design pattern 1
		\item Design pattern 2
		\item Design pattern 3
	\end{itemize}
	Being a prototype, and a university project, the system will not implement a payment system nor will the login/authentication microservice require any more than a login and password. No personal details will be requested.
	\pagebreak
	\section{Chapter 2: Literary Review}
	\subsection{Microservice}
	blah
	\subsection{Domain Driven Design}
	blah
	\subsection{Single-Responsibility Principle}
	blah
	\subsection{RESTful}
	blah
		\subsubsection{Communication}
		blah blah
		\subsubsection{API's}
		blah blah
		\subsubsection{CRUD Service}
		blah blah
	\subsection{polyglot Programming}
	blah
	
	\pagebreak
	\section{Chapter 3: Proposed Design}
	\subsection{Test}
	blah blah blah
	\pagebreak
	\section{Chapter 4: Prototype}
	\subsection{Analysis}
	\subsection{Requirement Specification}
	The following Requirement Specification was devised form the analysis:
	\subsubsection{Purpose}
The purpose of this document is too detail the requirements of the project's Web system Prototype. The Requirement specification contains the following headlines:
\begin{itemize}
	\item Project Brief - provides a description of the web system prototype. %as there is no "client" this may be the best option atm.
	\item Scope - details the size of the project
	\item Resources - What suitable resources have been sourced to implement the prototype 
	\item User Requirements - Provides a description of the services provided to users
	\item System Overview - contains an overview of what the system will do
	\item System Architecture - provides a high-level understanding of the design structure of the web system
	\item Functional Requirements - Details what each Microservices function
	\item Non-Functional Requirements - details the design of the front-end web User Interface (UI)
	\item Performance - contains the expected performance of the system
\end{itemize}
\subsubsection{Project Brief}
The prototype to be created will be an E-commerce web system. The web system will allow users to perform actions commonly associated with E-commerce web systems.

This prototype will provide basic functions commonly found in E-commerce websites:
\begin{itemize}
	\item A simple authentication service in the form of register and login
	\item Search for specific catalogue items
	\item Add items to Basket 
	\item manipulate items in the basket - edit/delete items
	\item Deal of the week
\end{itemize}
These functions will form the bounded context of Microservices within this prototype.

Due to this being a prototype system and the constraints of being a university student, in terms of confidentiality and data protection, this prototype will not be providing a means to buy items. And the register function will only ask for a user name and password to be created by the user. no personal data will be requested.
\subsubsection{Scope}
The web system will allow users to register a login name and password to use the features of the site. The catalogue available for browsing will be limited to a few items under the following categories:
\begin{itemize}
	\item clothes 
	\item toys  
	\item Electronics 
	\item books 
	\item miscellaneous
\end{itemize}
Users will also be able to add more than one item to their basket, selecting the quantity of each item they want to add.

The web system prototype must be usable for all major web browsers. The User Interface (UI) will be a simple and non-complex design so users can use it successfully.

This prototype will be finished by -Insert date xx/xx/2019- 
\subsubsection{Resources}
This project will use the following development software:
\begin{itemize}
	\item Microsoft Visual Studio - to develop the web system prototype using asp.net
	\item Programming code - C\# will be used to write the coding of the software and SQL will be used for all database query command
	\item Microsoft Visio - To create the UML diagrams used for the Analysis \& Design stages
	\item TeXstudio - Used to write up any documents required
	\item SQLite/Entity Framework - this will be used to store necessary data permanently
\end{itemize}
\subsubsection{User Requirement}

\subsubsection{System Overview}
The project will a web system that can run on the main web browsers available: Mozilla Firefox, Google Chrome, Microsoft Edge, Opera, Safari.

The Web system will provide several web pages that provide the user with different information and functionality depending on that web pages requirements.

Each Microservice will perform a single function that is derived from the business domain and bound within the context of that domain.

To simulate a distributed system, the web system and some microservices will be separated and treated as a client-server set up. Where by the Microservice(s) will be communicated with over the local host of the machine used for deployment.

Visual Studio will be used to develop the web system. 

The business model, databases, will be created using the Entity framework or SQLite. And each Microservice will have its own database that it is solely responsible.
\subsubsection{System Architecture}
The web system will incorporate the following architectures and design principles:
	\begin{itemize}\itemsep0pt
	\item \textbf{Microservice Architecture}
	\\This will be used for the high-level design of the website.
	\item \textbf{Model-View-Controller (MVC)}
	\\This layered Architecture style will be used within the front-end user Interface.This layered Architecture will also be the entry point to access the API gateway to access the required Microservice.
	\item \textbf{RESTful API}
	\\Used within the MVC architecture to provide the communication between the UI and Microservices in regards to database usage. Implementing GET,PUT,DELETE and POST HTTP methodologies.
	\item \textbf{API Gateway Pattern}
	\\Used to provide the access for the front-end User Interface to the individual Microservices available. The API gateway will provide the methods needed to access the required Microservice.
	\item \textbf{Domain Driven Design (DDD) Principle}
	\\ The model layer will be created using this principle so each model will be constructed to contain the business domain it was constructed for and each model will have its bounded context.
	\item \textbf{Single Responsibility Principle}
	\\ Will coincide with the above principle to be enforced and ensures each Microservice has only one main function to perform and, thusly, will only require minor changes for future expansion/maintenance
	\end{itemize}
\subsubsection{Functional Requirements}
The functional requirements will display the microservices to be developed and the functionality associated with each.
The web system will provide the following Microservices:
\begin{itemize}
	\item Authentication service
	\item Catalogue service
	\item Search Service
	\item Basket service
	\item payment service?
	\item Administrator Service
	\item Customer service service
	\item customer service
\end{itemize}
And provide the following functionality:
	\paragraph{Authentication Service}
		\begin{itemize}
			\item Provide users the ability to register
			\item Provide users the ability to login
			\item identify type of user logging in - customer or administrator
		\end{itemize}
	\paragraph{Catalogue Service}
		\begin{itemize}
			\item Fill the home page with a selection of available items
			\item Provide a detailed description of each item selected by a user
			\item Allow the user to select more than one of an item to buy
			\item Allow users to add items to their list
		\end{itemize}
	\paragraph{Search Service}
		\begin{itemize}
			\item Allow users to search for specific items from the catalogue
		\end{itemize}
	\paragraph{Basket Service}
		\begin{itemize}
			\item Allow users to view the items in their "basket"
			\item Allow users to edit their basket - delete items and increase/decrease the number ordered of individual items
		\end{itemize}
	\paragraph{Payment Service}
		\begin{itemize}
			\item Allows users to view their items they wish to buy
			\item Provide Users with options on how they want items delivered
			\item Provide Users with multiple Payment options
			\item Display an invoice of the order made
			\item send an email confirmation of the order to the customer
		\end{itemize}
	\paragraph{Administrator service}
	\paragraph{Customer service service}
	\paragraph{Customer service}
		\begin{itemize}
			\item Provide Customers the ability to view previous orders
			\item Provide Customers the ability to view lists they have created
			\item Allow uses to change their password if needed
		\end{itemize}
\subsubsection{Non-Functional Requirements}
\subsubsection{Performance}
\subsubsection{Glossary}

	\subsection{Design}
	\subsection{Development}
	\subsection{Testing}
	\subsection{Evaluation}

	
	\pagebreak
	\section{Chapter 5: Conclusion/Evaluation}
	\pagebreak
	\bibliography{Bibliography}
	\bibliographystyle{plain}
	\pagebreak
	%example of References. See https://en.wikibooks.org/wiki/LaTeX/Bibliography_Management
	%might be good to use a separate document for these so your main work is not one really long text file. 
	
	%you can crate this on a extra Tex document just like the title or any other part of the document.
	\newpage
	\begin{appendices}
		\section{Project Initiation Document}
		%insert IPO
		
		\begin{subappendices}
			\subsection{Example sub appendices}
			...
		\end{subappendices}
		
		\section{Second Formal Review Output}
		Insert a copy of the project review form you were given at the end of the review by the second marker
		
		\section{Diary Sheets (or other project management evidence)}
		Insert diary sheets here together with any project management plan you have
		\pagebreak
		\section{Class Diagrams}
		\begin{subappendices}
			\subsection{Authentication Microservice Classes}
				\begin{figure}[h]
					\caption{Authentication Classes}
					\label{fig:AuthentClass}
					\includegraphics[height=5cm, width=15cm]{AuthenticationMicroserviceClassDiagram}
				\end{figure}
		\subsection{Basket Microservice Classes}
			\begin{figure}[h]
				\caption{Basket Classes}
				\label{fig:BaskClass}
				\includegraphics[height=5cm, width=15cm]{BasketMicroserviceClassDiagram}
			\end{figure}
		\pagebreak
		\subsection{Catalogue Microservice Classes}
			\begin{figure}[h]
				\caption{Catalogue Classes}
				\label{fig:CatalogueClass}
				\includegraphics[height=10cm, width=15cm]{CatalogueMicroserviceClassDiagram}
			\end{figure}
		\pagebreak
		\subsection{Checkout Microservice Classes}
			\begin{figure}[h]
				\caption{Checkout Classes}
				\label{fig:CheckoutClass}
				\includegraphics[height=5cm, width=15cm]{CheckoutMicroserviceClassDiagram}
			\end{figure}
		\subsection{Customer Microservice Classes}
			\begin{figure}[h]
				\caption{Customer Classes}
				\label{fig:CustomerClass}
				\includegraphics[height=8cm, width=15cm]{CustomerMicroserviceClassDiagram}
			\end{figure}
		\end{subappendices}
	\end{appendices}
	
\end{document}